%% The MIT License (MIT)
%%
%% Copyright (c) 2015 Daniil Belyakov
%%
%% Permission is hereby granted, free of charge, to any person obtaining a copy
%% of this software and associated documentation files (the "Software"), to deal
%% in the Software without restriction, including without limitation the rights
%% to use, copy, modify, merge, publish, distribute, sublicense, and/or sell
%% copies of the Software, and to permit persons to whom the Software is
%% furnished to do so, subject to the following conditions:
%%
%% The above copyright notice and this permission notice shall be included in all
%% copies or substantial portions of the Software.
%%
%% THE SOFTWARE IS PROVIDED "AS IS", WITHOUT WARRANTY OF ANY KIND, EXPRESS OR
%% IMPLIED, INCLUDING BUT NOT LIMITED TO THE WARRANTIES OF MERCHANTABILITY,
%% FITNESS FOR A PARTICULAR PURPOSE AND NONINFRINGEMENT. IN NO EVENT SHALL THE
%% AUTHORS OR COPYRIGHT HOLDERS BE LIABLE FOR ANY CLAIM, DAMAGES OR OTHER
%% LIABILITY, WHETHER IN AN ACTION OF CONTRACT, TORT OR OTHERWISE, ARISING FROM,
%% OUT OF OR IN CONNECTION WITH THE SOFTWARE OR THE USE OR OTHER DEALINGS IN THE
%% SOFTWARE.

% The font could be set to Windows-specific Calibri by using the 'calibri' option
\documentclass[]{mcdowellcv}

% For mathematical symbols
\usepackage{amsmath}

% for links
\usepackage[colorlinks=true]{hyperref}

% \hypersetup{
%  pdfauthor={Alex Saltzman},
%  pdftitle={resume},
%  pdfkeywords={},
%  pdfsubject={},
%  pdflang={English},
%  colorlinks,
%  linkcolor={black},
%  citecolor={black},
%  urlcolor={blue}
% }

% fancy lists
\usepackage{enumitem}



% Set applicant's personal data for header
\name{Alexander B. Saltzman}
\address{Houston, TX 77054}
\contacts{(281) 433-8545 \linebreak a.saltzman920@gmail.com}

\begin{document}

	% Print the header
	\makeheader

	% Print the content

  % =============================================================================

	\begin{cvsection}{Education}
		\begin{cvsubsection}{2014 -- Present}{Ph.D. Biochemistry}{Baylor College of Medicine, Houston TX}
      advisor: Anna Malovannaya, Ph.D.
			\begin{itemize}
        \item Thesis project on data generation and analysis of proteomic data.
        \item Built and published \texttt{gpGrouper}, an alternative peptide
          grouping program that considers mixed species patient-derived
          xenografts. \href{https://www.ncbi.nlm.nih.gov/pubmed/30093420}{PMID 30093420}
        \item Built automated pipelines for analyzing mass spectrometry data.
        \item Performed quality-control and exploratory data analysis.
        \item Effectively summarized and visualized thousands of datapoints.
        \item Trained technicians and other scientists on software usage.
        \item Collaborated with other scientists on proper statistical analysis
          and modeling.
			\end{itemize}
		\end{cvsubsection}

    \begin{cvsubsection}{2009 -- 2013}{BS, Biochemistry}{Texas A\&M University, College Station TX}
			\begin{itemize}
        \item Magna Cum Laude, GPA 3.7/4.0
        \item \textbf{Undergraduate Research Scholar} : Competed a year-long
          research program with an initial research proposal, primary laboratory
          work, written thesis, and participation in a judged poster session.
      \end{itemize}
		\end{cvsubsection}
	\end{cvsection}

  % =============================================================================

	\begin{cvsection}{Languages and Technologies}
		\begin{cvsubsection}{}{}{}
      {\bfseries Proficient In}
      \begin{description}[style=multiline,align=left,labelindent=2em]
        \item [languages] Python, R, \LaTeX, HTML, SQL
				\item [technologies] Matplotlib, Numpy, Scipy, Pandas, scikit-learn,
          Seaborn, plot.py, Dash, ggplot, tidyverse, R Shiny, Bash Scripting,
          Git, Vim, Emacs, Ubuntu Linux, OSX
			\end{description}
		\end{cvsubsection}
		\begin{cvsubsection}{}{}{}
      {\bfseries Have Experience With}
      \begin{description}[style=multiline,align=left,labelindent=2em]
        \item [languages] MATLAB, Javascript,
        \item [technologies] Django, Flask, Postgres, SQLite, unit testing,
          Jinja, Nginx, Bootstrap
			\end{description}
		\end{cvsubsection}
	\end{cvsection}

  % =============================================================================

	\begin{cvsection}{Selected Projects}
		% \begin{cvsubsection}{Projects}{}{}
    \begin{cvsubsection}{}{}{}

			% \begin{itemize}
			% \begin{labeling}{BMB Retreat }
			% \begin{description}[leftmargin=8em,style=nextline,align=parleft]
			\begin{description}[leftmargin=8em,style=multiline,align=parleft,labelwidth=!]

      \item [\textbf{gpGrouper}] Protein inference software for bottom-up
        proteomics with shared peptide peak area redistribution.
        \href{https://github.com/asalt/gpgrouper}{GitHub: asalt/gpgrouper}.
      \item [\textbf{RefProtDB}] Program for downloading protein sequences from
        NCBI and condensing protein isoforms under a single gene identifier.
        \href{https://github.com/asalt/RefProtDB}{GitHub: asalt/RefProtDB}.
      \item [\textbf{BMB Retreat Website }] Built a sign-up and abstract
        submission website in \texttt{Django} for the annual departmental
        retreat.
      \vspace*{1em} % make room for long label
      \item [\textbf{tackle}] A CLI for downloading and plotting (scatter, PCA,
        clustered heatmap) proteomics data stored at Baylor College of Medicine.
      \item [\textbf{autogrouper}] A wrapper around \texttt{gpGrouper} that
        automates data processing via connection to network proteomics database.
      \item [\textbf{bcmproteomics}] A Python module for connecting to network
        proteomics database via PyODBC. Provides containers for loading and
        saving metadata and results.
      \item [\textbf{filegrouper}] CLI for sorting files from one folder into
        separate folders based on common names.
        \href{https://github.com/asalt/filegrouper}{GitHub: asalt/filegrouper}.

			\end{description}

    \end{cvsubsection}
		% \end{cvsubsection}
	\end{cvsection}

  % =============================================================================
  \vspace*{6em} % so that it starts on next page
	\begin{cvsection}{Teaching and Leadership}
		\begin{cvsubsection}{2016 -- 2018}{Graduate Student Council
        Representative}{Baylor College of Medicine}
      \vspace*{1em} % make room for long label
			\begin{itemize}
        \item Organize events for graduate students including town halls,
          fundraisers, and annual student symposium.
        \item \textbf{Career Development Committee Head Rep :} Help organize
          seminars with speakers on various post-graduate careers.
          \begin{itemize}
            \item Send invitations and coordinate speaker arrivals.
          \end{itemize}
        \end{itemize}

    \end{cvsubsection}

    \begin{cvsubsection}{}{}{}
    \begin{description}[leftmargin=12em,style=multiline,align=parleft,labelwidth=!]
      \item [Invited Reviewer : Thinking Like A Scientist Course] Lead a small
        group in their analysis and presentation of a research paper.
        \textit{Baylor College of Medicine : 2016} \vspace*{2em}
      \item [First Year Graduate Student Mentor] Offered coursework and choosing
        a mentor for a first year graduate student.
        \textit{Baylor College of Medicine : 2015 -- 2016} \vspace*{1em}
      \item [High School Tutor] Tutored students for the Texas Assessment of
        Knowledge and Skills (TAKS) standardized math test.
        \textit{Clear Lake High School : 2012}

    \end{description}
    \end{cvsubsection}

	\end{cvsection}

  % =============================================================================

	\begin{cvsection}{Employment}

		\begin{cvsubsection}{Jan -- July 2014}{Math Tutor}{San Jacinto Community College Student Success Center, Pasadena TX}
      \begin{itemize}
      \item Help students with homework and key concepts in a variety of topics.
      \item Algebra, geometry, trigonometry, calculus, differential equations
      \item Chemistry, physics, biology.
      \end{itemize}
    \end{cvsubsection}

		\begin{cvsubsection}{2010--2013}{Student Researcher}{Texas A\&M Health Science Center, College Station TX}
      \vspace*{2em} % make room for long subsection title
			\begin{itemize}
        \item Prepared buffer solutions, washed and autoclaved glassware,
          stocked and organized reagents, filled out spending spreadsheets.
        \item Assisted researches taking measurements and images in the mouse facility.
			\end{itemize}
		\end{cvsubsection}

	\end{cvsection}

  % =============================================================================


\end{document}
